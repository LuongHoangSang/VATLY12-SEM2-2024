\def\linkdethi{https://toanvip307.blogspot.com/} %Đường kinh cần dẫn tới
\hienmaQR
%%%%%%%%%%%%%%%%%%%%%%%% Tiêu đề 3 tham số
\begin{dethi}
	{Huyện Bình Chánh}
	{ĐỀ THAM KHẢO}
	{Kỳ thi tuyển sinh vào lớp 10}
\end{dethi}
\setcounter{EX}{0}  
%% Bài 1
\begin{ex}[1,5 điểm]%[Đề tham khảo tuyển sinh vào 10 - Quận Bình Chánh - Đề số 1]%[9D4-2]%[TS10B1]
	Cho $( P) : y= x^2$ và $(d) : y=-x+2$
	\begin{enumerate}
		\item Vẽ đồ thị $(P)$ và $(d)$ trên cùng một mặt phẳng tọa độ $O x y$.
		\item Tìm tọa độ giao điểm của $( P)$ và $( d)$ bằng phép toán.
	\end{enumerate}
	\loigiai{
		\begin{listEX}
			\item \immini
			{
				Bảng giá trị\\*[0.5 cm]
				\begin{tabular}{|c|c|c|}
					\hline
					$x$ & $0$ & $1$ \\
					\hline
					$y=-x+2$ & $2$ & $1$ \\
					\hline
				\end{tabular}
				\\*[0.5 cm]
				\begin{tabular}{|c|c|c|c|c|c|}
					\hline
					$x$ & $-2$ & $-1$ & $0$ & $1$ & $2$ \\
					\hline
					$y=x^2$ & $4$ & $1$ & $0$ & $1$ & $4$ \\
					\hline
				\end{tabular}
			}
			{
				\begin{tikzpicture}[>=stealth,x=1cm,y=1cm,scale=0.8]
					\draw[->] (-6,0) -- (6,0) node[below] {$x$};
					\draw[->] (0,-2) -- (0,6) node[left] {$y$};
					\draw (0,0)node[above left]{$O$};
					\draw[thick,samples=150,smooth,domain=-2.1:2.1] plot(\x,{(\x)^2})node[above right]{$y=x^2$};
					\draw[thick,samples=150,smooth,domain=-2.2:2.2] plot(\x,{(-1*\x)+2}) node[below right]{$y=-x+2$};
					\draw[dashed]
					(-2,0) node[below]{$-2$}--(-2,4)--(0,4) node[below left]{$-2$}--(2,4)--(2,0) node[above]{$2$}
					(0,-1) node[right]{$-1$}
					(-1,0) node[below]{$-1$}--(-1,1)--(0,1) node[above left]{$1$}--(1,1)--(1,0) node[below]{$1$}
					;
				\end{tikzpicture}
			}
			\item Phương trình hoành độ giao điểm của $(P)$ và $(d)$:
			\begin{eqnarray*}
				&x^2 &=-x+2\\
				\Leftrightarrow &x^2+x-2&=0\\
				\Leftrightarrow &x=1 ~\text{hay} ~x&=-2
			\end{eqnarray*}
			\begin{itemize}
				\item $x=1  \Rightarrow y =1^2=1$.
				\item $x=-2 \Rightarrow y =(-2)^2=4$.
			\end{itemize}
			Vậy tọa độ giao điểm của $(P)$ và $(d)$ là $(1;1)$ , $(-2;4)$.
		\end{listEX}
	}
\end{ex}

%%%%%%%%%%%%%%
\label{de\thedeso} %nhãn đề
