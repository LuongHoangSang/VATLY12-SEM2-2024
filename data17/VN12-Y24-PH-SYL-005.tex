\let\lesson\undefined
\newcommand{\lesson}{\phantomlesson{Bài 4: Nhiệt dung riêng, nhiệt nóng chảy riêng, nhiệt hoá hơi riêng}}
\chapter[Nhiệt nóng chảy riêng. Nhiệt hoá hơi riêng]{Nhiệt nóng chảy riêng. Nhiệt hoá hơi riêng}
\section{Lý thuyết}
\subsection{Nhiệt nóng chảy riêng}
Nhiệt nóng chảy riêng của một chất rắn có giá trị bằng nhiệt lượng cần cung cấp cho $\SI{1}{\kilogram}$ chất đó chuyển từ thể rắn sang thể lỏng tại nhiệt độ nóng chảy:
\begin{equation}
	\lambda=\dfrac{Q}{m}
\end{equation}
với
\begin{itemize}
	\item $\lambda$: nhiệt nóng chảy riêng, đơn vị trong hệ SI là $\si{\joule/\kilogram}$;
	\item $Q$: nhiệt lượng khối chất rắn thu vào để nóng chảy hoàn toàn, đơn vị trong hệ SI là $\si{\joule}$;
	\item $m$: khối lượng của khối chất rắn, đơn vị trong hệ SI là $\si{\kilogram}$.
\end{itemize}
\subsection{Nhiệt hoá hơi riêng}
Nhiệt hoá hơi riêng của một chất lỏng có giá trị bằng nhiệt lượng cần cung cấp cho $\SI{1}{\kilogram}$ chất lỏng đó hoá hơi hoàn toàn ở nhiệt độ sôi:
\begin{equation}
	L=\dfrac{Q}{m}
\end{equation}
với
\begin{itemize}
	\item $L$: nhiệt hoá hơi riêng, đơn vị trong hệ SI là $\si{\joule/\kilogram}$;
	\item $Q$: nhiệt lượng khối chất lỏng thu vào để hoá hơi hoàn toàn, đơn vị trong hệ SI là $\si{\joule}$;
	\item $m$: khối lượng của khối chất lỏng, đơn vị trong hệ SI là $\si{\kilogram}$.
\end{itemize}
\section{Mục tiêu bài học - Ví dụ minh hoạ}
\begin{dang}{Vận dụng biểu thức xác định nhiệt nóng chảy riêng}
	\viduii{2}
	{Một nhà máy thép mỗi lần luyện được 35 tấn thép. Cho nhiệt nóng chảy riêng của thép là $\SI{2.77E5}{\joule/\kilogram}$.
		\begin{enumerate}[label=\alph*)]
			\item Tính nhiệt lượng cần cung cấp để làm nóng chảy thép trong mỗi lần luyện của nhà máy ở nhiệt độ nóng chảy.
			\item Giả sử nhà máy sử dụng khí đốt để nấu chảy thép trong lò thổi (nồi nấu thép). Biết khi đốt cháy hoàn toàn $\SI{1}{\kilogram}$ khí đốt thì nhiệt lượng toả ra là $\SI{44E6}{\joule}$. Xác định khối lượng khí đốt cần sử dụng.
		\end{enumerate}
	
}
{\hide{
	\begin{enumerate}[label=\alph*)]
		\item Nhiệt lượng cần cung cấp để làm nóng chảy thép trong mỗi lần luyện của nhà máy ở nhiệt độ nóng chảy:
		$$Q=m\lambda=\left(\SI{35E3}{\kilogram}\right)\cdot\left(\SI{2.77E5}{\joule/\kilogram}\right)=\SI{96.95E8}{\joule}.$$
		\item Khối lượng khí đốt cần sử dụng để nhiệt lượng toả ra như ở câu a):
		$$m=\dfrac{Q}{q}=\dfrac{\SI{96.95E8}{\joule}}{\SI{44E6}{\joule/\kilogram}}\approx\SI{220.34}{\kilogram}.$$
	\end{enumerate}}
}

\viduii{3}
{Tính thời gian cần thiết để làm nóng chảy hoàn toàn $\SI{2}{\kilogram}$ đồng có nhiệt độ ban đầu $\SI{30}{\celsius}$, trong một lò nung điện có công suất $\SI{20000}{\watt}$. Biết chỉ có $\SI{50}{\percent}$ năng lượng tiêu thụ của lò được dùng vào việc làm đồng nóng lên và nóng chảy hoàn toàn ở nhiệt độ không đổi. Biết nhiệt độ nóng chảy của đồng là $\SI{1084}{\celsius}$. Cho nhiệt dung riêng, nhiệt nóng chảy riêng của đồng lần lượt là $\SI{380}{\joule/\left(\kilogram\cdot\kelvin\right)}$ và $\SI{1.8E5}{\joule/\kilogram}$.

}
{\hide{
	Nhiệt lượng khối đồng cần thu vào để tăng nhiệt độ từ $\SI{30}{\celsius}$ đến $\SI{1084}{\celsius}$:
	$$Q_1=mc\Delta t=\left(\SI{2}{\kilogram}\right)\cdot\left[\SI{380}{\joule/\left(\kilogram\cdot\kelvin\right)}\right]\cdot\left(\SI{1084}{\celsius}-\SI{30}{\celsius}\right)=\SI{801.04}{\kilo\joule}.$$
	Nhiệt lượng khối đồng cần thu vào để nóng chảy hoàn toàn ở nhiệt độ $\SI{1084}{\celsius}$:
	$$Q_2=m\lambda=\left(\SI{2}{\kilogram}\right)\cdot\left(\SI{1.8E5}{\joule/\kilogram}\right)=\SI{360}{\kilo\joule}.$$
	Tổng nhiệt lượng $\SI{2}{\kilogram}$ đồng cần thu vào để nóng chảy hoàn toàn từ nhiệt độ ban đầu $\SI{30}{\celsius}$:
	$$Q=Q_1+Q_2=\SI{1161.04}{\kilo\joule}.$$
	Thời gian cần thiết để làm nóng chảy hoàn toàn khối đồng này:
	$$t=\dfrac{Q}{H\cdot\calP}=\dfrac{\SI{1161.04E3}{\joule}}{\left(\SI{50}{\percent}\right)\cdot\left(\SI{2E4}{\watt}\right)}\approx\SI{116}{\second}.$$
}
}

\end{dang}
\begin{dang}{Vận dụng biểu thức xác định nhiệt hoá hơi riêng}
	\viduii{2}
	{Tính nhiệt lượng cần thiết để làm cho $\SI{1}{\kilogram}$ nước ở $\SI{25}{\celsius}$ chuyển thành hơi ở $\SI{100}{\celsius}$. Cho nhiệt dung riêng của nước là $\SI{4200}{\joule/\left(\kilogram\cdot\kelvin\right)}$, nhiệt hoá hơi riêng của nước ở $\SI{100}{\celsius}$ là $\SI{2.26E6}{\joule/\kilogram}$.
	
}
{\hide{
	Nhiệt lượng nước thu vào để tăng nhiệt độ từ $\SI{25}{\celsius}$ đến $\SI{100}{\celsius}$:
	$$Q_1=mc\Delta t=\left(\SI{10}{\kilogram}\right)\cdot\left[\SI{4200}{\joule/\left(\kilogram\cdot\kelvin\right)}\right]\cdot\left(\SI{100}{\celsius}-\SI{25}{\celsius}\right)=\SI{315E4}{\joule}.$$
	Nhiệt lượng nước cần thu vào để hoá thành hơi hoàn toàn ở $\SI{100}{\celsius}$:
	$$Q_2=mL=\left(\SI{10}{\kilogram}\right)\cdot\left(\SI{2.26E6}{\joule/\kilogram}\right)=\SI{2260E4}{\joule}.$$
	Tổng nhiệt lượng nước cần thu vào để hoá hơi hoàn toàn ở $\SI{100}{\celsius}$:
	$$Q=Q_1+Q_2=\SI{25.75}{\mega\joule}.$$
}
}


\viduii{3}
{Một ấm đun nước có công suất $\SI{500}{\watt}$ chứa $\SI{300}{\gram}$ nước ở nhiệt độ $\SI{20}{\celsius}$. Cho nhiệt dung riêng của nước là $\SI{4200}{\joule/\left(\kilogram\cdot\kelvin\right)}$, nhiệt hoá hơi riêng của nước ở $\SI{100}{\celsius}$ là $\SI{2.26E6}{\joule/\kilogram}$.
	\begin{enumerate}[label=\alph*)]
		\item Tính thời gian cần thiết để đun nước trong ấm để đạt đến nhiệt độ sôi.
		\item Sau khi nước đến nhiệt độ sôi, người ta để ấm tiếp tục đun nước sôi trong 2 phút. Tính khối lượng nước còn lại trong ấm và chỉ rõ điều kiện để thực hiện các tính toán đó.
	\end{enumerate}

}
{\hide{
	\begin{enumerate}[label=\alph*)]
		\item Nhiệt lượng nước trong ấm cần thu vào để tăng nhiệt độ từ $\SI{20}{\celsius}$ đến nhiệt độ sôi $\left(\SI{100}{\celsius}\right)$:
		$$Q_1=mc\Delta t=\left(\SI{0.3}{\kilogram}\right)\cdot\left[\SI{4200}{\joule/\left(\kilogram\cdot\kelvin\right)}\right]\cdot\left(\SI{100}{\celsius}-\SI{20}{\celsius}\right)=\SI{100800}{\joule}.$$
		Thời gian đun sôi nước:
		$$t=\dfrac{Q_1}{\calP}=\dfrac{\SI{100800}{\joule}}{\SI{500}{\watt}}\approx\SI{201}{\second}.$$
		\item Nhiệt lượng ấm toả ra trong 2 phút:
		$$Q_2=\calP\cdot t'=\left(\SI{500}{\watt}\right)\cdot\left(\SI{120}{\second}\right)=\SI{60}{\kilo\joule}.$$
		Khối lượng nước bị hoá thành hơi ở nhiệt độ $\SI{100}{\celsius}$:
		$$m'=\dfrac{Q_2}{L}=\dfrac{\SI{60E3}{\joule}}{\SI{2.26E6}{\joule/\kilogram}}\approx\SI{26.55}{\gram}.$$
		Khối lượng nước còn lại trong ấm:
		$$m_\text{nước}=m-m'=\SI{273.45}{\gram}.$$
		Các tính toán trên được thực hiện với điều kiện:
		\begin{itemize}
			\item Nước được đun ở áp suất $\SI{1}{atm}$, do đó nhiệt độ sôi của nước là $\SI{100}{\celsius}$.
			\item Bỏ qua nhiệt lượng cung cấp cho vỏ ấm đun và toả ra môi trường.
			\item Bỏ qua sự bay hơi của nước trong quá trình đun.
		\end{itemize}
	\end{enumerate}
}
}	
\end{dang}
