\section{ÔN TẬP CHƯƠNG I}
\subsection{Câu trắc nghiệm nhiều phương án lựa chọn}
\textit{Thí sinh trả lời từ câu 1 đến câu 18. Mỗi câu hỏi thí sinh chọn một phương án}
\Opensolutionfile{ans}[ans/G12Y24B7TN]
% ===================================================================
\begin{ex}
	Quy ước dấu nào sau đây phù hợp với định luật I của nhiệt động lực học?
	\choice
	{Vật nhận công $A<0$; vật nhận nhiệt $Q<0$}
	{Vật thực hiện công $A>0$; vật truyền nhiệt lượng $Q<0$}
	{\True Vật nhận công $A>0$; vật nhận nhiệt lượng $Q>0$}
	{Vật thực hiện công $A>0$; vật truyền nhiệt lượng $Q>0$}
	\loigiai{}
\end{ex}
% ===================================================================
\begin{ex}
	Ở nhiệt độ phòng, chất nào sau đây không tồn tại ở thể lỏng?
	\choice
	{Rượu}
	{\True Nhôm}
	{Thuỷ ngân}
	{Nước}
	\loigiai{}
\end{ex}
% ===================================================================
\begin{ex}
	Vật nào sau đây có cấu trúc tinh thể?
	\choice
	{\True Chiếc cốc thuỷ tinh}
	{Hạt muối ăn}
	{Viên kim cương}
	{Miếng thạch anh}
	\loigiai{}
\end{ex}
% ===================================================================
\begin{ex}
Điều nào sau đây là \textbf{sai} khi nói về sự đông đặc?	
	\choice
	{Sự đông đặc là quá trình chuyển từ thể lỏng sang thể rắn}
	{\True Với một chất rắn, nhiệt độ đông đặc luôn nhỏ hơn nhiệt độ nóng chảy}
	{Trong suốt quá trình đông đặc, nhiệt độ của vật không thay đổi}
	{Nhiệt độ đông đặc của các chất thay đổi theo áp suất bên ngoài}
	\loigiai{}
\end{ex}
% ===================================================================
\begin{ex}
	Biểu thức nào sau đây là biểu thức chuyển đổi đúng đơn vị nhiệt độ từ $\si{\celsius}$ sang thang $\si{\kelvin}$?
	\choice
	{\True $\xsi{T}{\left(\si{\kelvin}\right)}=\xsi{t}{\left(\si{\celsius}\right)}+273$}
	{$\xsi{T}{\left(\si{\kelvin}\right)}=\xsi{t}{\left(\si{\celsius}\right)}-273$}
	{$\xsi{T}{\left(\si{\kelvin}\right)}=\dfrac{9}{5}\xsi{t}{\left(\si{\celsius}\right)}+273$}
	{$\xsi{T}{\left(\si{\kelvin}\right)}=\dfrac{9}{5}\xsi{t}{\left(\si{\celsius}\right)}-273$}
	\loigiai{}
\end{ex}
% ===================================================================
\begin{ex}
	Trong thang nhiệt độ Celsius, nhiệt độ không tuyệt đối là
	\choice
	{$\SI{100}{\celsius}$}
	{\True $\SI{-273}{\celsius}$}
	{$\SI{0}{\celsius}$}
	{$\SI{-32}{\celsius}$}
	\loigiai{}
\end{ex}
% ===================================================================
\begin{ex}
	Đơn vị nhiệt nóng chảy riêng của vật rắn là
	\choice
	{$\si{\joule}$}
	{$\si{\joule/\kelvin}$}
	{\True $\si{\joule/\kilogram}$}
	{$\si{\joule/\left(\kilogram\cdot\kelvin\right)}$}
	\loigiai{}
\end{ex}

% ===================================================================
\begin{ex}
	Kết luận nào sau đây \textbf{không đúng} với thang nhiệt độ Celsius?
	\choice
	{Đơn vị đo nhiệt độ là $\si{\celsius}$}
	{Chọn mốc nhiệt độ nước đá đang tan ở áp suất $\SI{1}{atm}$ là $\SI{0}{\celsius}$}
	{Chọn mốc nhiệt độ nước sôi ở áp suất $\SI{1}{atm}$ là $\SI{100}{\celsius}$}
	{\True $\SI{1}{\celsius}$ tương ứng với $\SI{273}{\kelvin}$}
	\loigiai{}
\end{ex}
% ===================================================================
\begin{ex}
	Nhiệt hoá hơi riêng của nước là $\SI{2.3E6}{\joule/\kilogram}$. Phát biểu nào dưới đây là \textbf{đúng}?
	\choice
	{\True Mỗi kilogram nước cần thu một lượng nhiệt $\SI{2.3E6}{\joule}$ để bay hơi hoàn toàn ở nhiệt độ sôi và áp suất chuẩn}
	{Mỗi kilogram nước cần thu một lượng nhiệt $\SI{2.3E6}{\joule}$ để bay hơi hoàn toàn}
	{Mỗi kilogram nước cần toả ra một lượng nhiệt $\SI{2.3E6}{\joule}$ để bay hơi hoàn toàn ở nhiệt độ sôi}
	{Một lượng nước bất kì cần thu một lượng nhiệt là $\SI{2.3E6}{\joule}$ để bay hơi hoàn toàn}
	\loigiai{}
\end{ex}
% ===================================================================
\begin{ex}
	Hãy tìm ý \textbf{không đúng} với mô hình động học phân tử.
	\choice
	{\True Tốc độ chuyển động của các phân tử cấu tạo nên vật càng lớn thì thể tích của vật càng lớn}
	{Các chất được cấu tạo từ các hạt riêng biệt gọi là phân tử}
	{Các phân tử chuyển động không ngừng}
	{Giữa các phân tử có lực tương tác gọi là lực liên kết phân tử}
	\loigiai{}
\end{ex}
% ===================================================================
\begin{ex}
	Điểm đóng băng và điểm sôi của nước theo thang nhiệt độ Kelvin là
	\choice
	{\True $\SI{273}{\kelvin}$ và $\SI{373}{\kelvin}$}
	{$\SI{0}{\kelvin}$ và $\SI{100}{\kelvin}$}
	{$\SI{73}{\kelvin}$ và $\SI{37}{\kelvin}$}
	{$\SI{32}{\kelvin}$ và $\SI{212}{\kelvin}$}
	\loigiai{}
\end{ex}
% ===================================================================
\begin{ex}
	Người ta cung cấp cho khí trong một cylanh nằm ngang nhiệt lượng $\SI{2}{\joule}$. Đồng thời nén piston một đoạn $\SI{5}{\centi\meter}$ với một lực có độ lớn là $\SI{20}{\newton}$. Độ biến thiên nội năng của khí là
	\choice
	{\True $\SI{3}{\joule}$}
	{$\SI{1}{\joule}$}
	{$\SI{-1}{\joule}$}
	{$\SI{-3}{\joule}$}
	\loigiai{$$\Delta U=Q+A=Q+F\cdot s=\SI{3}{\joule}.$$
	}
\end{ex}
% ===================================================================
\begin{ex}
	Nhiệt lượng cần cung cấp cho $\SI{0.5}{\kilogram}$ nước ở $\SI{0}{\celsius}$ đến khi nó sôi là bao nhiêu? Biết nhiệt dung riêng của nước là $\SI{4180}{\joule/\left(\kilogram\cdot\kelvin\right)}$.
	\choice
	{$\SI{5E5}{\joule}$}
	{$\SI{3E5}{\joule}$}
	{$\SI{4.18E5}{\joule}$}
	{\True $\SI{2.09E5}{\joule}$}
	\loigiai{$$Q=mc\Delta t=\SI{2.09E5}{\joule}.$$
	}
\end{ex}
% ===================================================================
\begin{ex}
	Biết nhiệt dung riêng của nước là $\SI{4200}{\joule/\left(\kilogram\cdot\kelvin\right)}$, nhiệt nóng chảy riêng của nước đá là $\SI{3.4E5}{\joule/\kilogram}$. Nhiệt lượng cần cung cấp cho $\SI{2}{\kilogram}$ nước đá ở nhiệt độ $\SI{0}{\celsius}$ là bao nhiêu để tăng nhiệt độ lên $\SI{60}{\celsius}$ là
	\choice
	{$\SI{0.72E6}{\joule}$}
	{\True $\SI{1.184E6}{\joule}$}
	{$\SI{2.254E6}{\joule}$}
	{$\SI{1.548E6}{\joule}$}
	\loigiai{Nhiệt lượng cần cung cấp cho $\SI{2}{\kilogram}$ nước đá ở nhiệt độ $\SI{0}{\celsius}$ là bao nhiêu để chuyển lên nhiệt độ $\SI{60}{\celsius}$ là
		$$Q=m\lambda+mc\Delta t=\SI{1.184E6}{\joule}.$$
	}
\end{ex}
% ===================================================================
\begin{ex}
Một lượng nước và một lượng rượu có thể tích bằng nhau, được cung cấp các nhiệt lượng tương ứng là $Q_1$ và $Q_2$. Biết khối lượng riêng của nước là $\SI{1000}{\kilogram/\meter^3}$ và của rượt là $\SI{800}{\kilogram/\meter^3}$, nhiệt dung riêng của nước và rượu lần lượt là $\SI{4200}{\joule/\left(\kilogram\cdot\kelvin\right)}$ và $\SI{2500}{\joule/\left(\kilogram\cdot\kelvin\right)}$. Để độ tăng nhiệt độ của nước và rượu bằng nhau thì
	\choice
	{$Q_1=Q_2$}
	{$Q_1=1,68Q_2$}
	{$Q_1=1,25Q_2$}
	{\True $Q_1=2,10Q_2$}
	\loigiai{$$\dfrac{Q_1}{Q_2}=\dfrac{m_1c_1}{m_2c_2}=\dfrac{\rho_1c_1}{\rho_2c_2}=2,1.$$
	}
\end{ex}
% ===================================================================
\begin{ex}
	Một ấm đun nước bằng nhôm có khối lượng $\SI{400}{\gram}$, chứa 3 lít nước được đun trên bếp. Khi nhận nhiệt lượng $\SI{740}{\kilo\joule}$ thì ấm đạt đến nhiệt độ $\SI{80}{\celsius}$. Biết nhiệt dung riêng của nhôm là $\SI{880}{\joule/\left(\kilogram\cdot\kelvin\right)}$, nhiệt dung riêng của nước là $\SI{4190}{\joule/\left(\kilogram\cdot\kelvin\right)}$. Coi nhiệt lượng mà ấm toả ra bên ngoài là không đáng kể. Nhiệt độ ban đầu của ấm và nước là
	\choice
	{$\SI{45.2}{\celsius}$}
	{\True $\SI{22.7}{\celsius}$}
	{$\SI{37.2}{\celsius}$}
	{$\SI{16.7}{\celsius}$}
	\loigiai{$$\Delta t=\dfrac{Q}{m_1c_1+m_2c_2}=\SI{57.27}{\celsius}\Rightarrow t_0=\SI{22.7}{\celsius}.$$
	}
\end{ex}
% ===================================================================
\begin{ex}
	Ở một số quốc gia, khi vận chuyển sữa trên xe tải, người ta sử dụng nitrogen lỏng thay vì tủ lạnh cơ học. Một chuyến giao hàng cần \SI{200}{\liter} nitrogen lỏng, với khối lượng riêng là \SI{808}{\kilogram/\meter^3}. Ban đầu nitrogen lỏng đang ở nhiệt độ sôi là \SI{-196}{\celsius} và khi đến địa điểm giao hàng thì nhiệt độ của nitrogen lỏng là \SI{3}{\celsius}. Nhiệt làm mát mà nitrogen lỏng cung cấp chính là lượng nhiệt cần thiết để làm bay hơi lượng nitrogen lỏng này và nâng nhiệt độ của nó lên đến \SI{3}{\celsius}. Biết rằng nhiệt dung riêng của khí nitrogen và nhiệt hoá hơi riêng của nitrogen lỏng lần lượt là \SI{1040}{\joule/\left(\kilogram\cdot\kelvin\right)} và \SI{199}{\kilo\joule/\kilogram}. Nhiệt lượng mà nitrogen lỏng nhận được trong quá trình này bằng
	\choice
	{\True \SI{65603.1}{\kilo\joule}}
	{\SI{32158.4}{\kilo\joule}}
	{\SI{33444.7}{\kilo\joule}}
	{\SI{64594.8}{\kilo\joule}}
	\loigiai{}
\end{ex}
% ===================================================================
\begin{ex}
	Một lượng khí khi bị nung nóng đã tăng thể tích thêm $\SI{0.02}{\meter^3}$ và nội năng biến thiên $\SI{1280}{\joule}$. Biết trong quá trình thay đổi thể tích thì áp suất khí luôn bằng $\SI{2E5}{\pascal}$. Nhiệt lượng đã truyền cho khí là
	\choice
	{$\SI{2720}{\joule}$}
	{$\SI{1280}{\joule}$}
	{\True $\SI{5280}{\joule}$}
	{$\SI{4000}{\joule}$}
	\loigiai{Công khối khí thực hiện trong quá trình dãn nở:
		$$A'=F\cdot \Delta x=pS\Delta x=p\Delta V=\SI{4000}{\joule}.$$
		Nhiệt lượng mà khí đã nhận:
		$$Q=\Delta U-A=\Delta U+A'=\SI{5280}{\joule}.$$
	}
\end{ex}

\Closesolutionfile{ans}
\subsection{CÂU TRẮC NGHIỆM ĐÚNG/SAI}
\setcounter{ex}{0}
\textit{Thí sinh trả lời từ câu 1 đến câu 4. Trong mỗi ý \textbf{a)}, \textbf{b)}, \textbf{c)}, \textbf{d)} ở mỗi câu, thí sinh chọn đúng hoặc sai.}
% ===================================================================
\begin{ex}
	\immini{Cho hình vẽ dưới đây, trong đó: (a) Khoảng cách và sự sắp xếp các phân tử ở các thể khác nhau; (b) Chuyển động của phân tử ở các thể khác nhau; Hình cầu là phân tử, mũi tên là hướng chuyển động của phân tử.
	}
	{\includegraphics[scale=0.4]{figs/G12Y24B7-4}}
	\choiceTF[t]
	{\True Hình 1a mô tả khoảng cách và sự sắp xếp của các phân tử ở thể khí}
	{Hình 2 b mô tả khoảng cách và sự sắp xếp của các phân tử ở thể lỏng}
	{Ở hình 2 , các phân tử sắp xếp không có trật tự, chặt chẽ}
	{\True Quá trình chuyển thể ở hình 3 sang thể ở hình 1 khi được đun nóng gọi là sự hóa hơi}
	\loigiai{}
\end{ex}
% ===================================================================
\begin{ex}
	\immini{Đốt nóng khối khí trong xi lanh đặt nằm ngang bằng ngọn lửa đèn cồn như hình vẽ. Khí giãn nở đẩy pít - tông từ vị trí (1) đến vị trí (2).}{\vspace{-0.5cm}\includegraphics[scale=0.3]{figs/G12Y24B7-3}}
	\choiceTF[t]
	{\True Khối khí trong xi lanh nhận nhiệt lượng $Q (Q>0)$}
	{Khí dãn nở và nhận công $A (A>0)$}
	{Nội năng của khối khí khi pít - tông ở vị trí (2) là $\Delta U=Q+A$ với $Q>0$ và $A<0$}
	{\True Khi khối khí trong xi lanh nhận được một nhiệt lượng \SI{150}{\joule} thì khối khí giãn nở làm thể tích tăng từ \SI{20}{\centi\meter^3} đến \SI{30}{\centi\meter^3}, biết rằng áp suất của khối khí trong xilanh không đổi và bằng \SI{5E5}{\pascal}. Nội năng của khối khí trong quá trình này tăng \SI{145}{\joule}}
	\loigiai{
	\begin{enumerate}[label=\alph*)]
		\item Đúng.
		\item Sai. Khí dãn nở và sinh công nên $A<0$.
		\item Sai. $\Delta U$ là độ biến thiên nội năng chứ không phải nội năng.
		\item Đúng.
	\end{enumerate}
	}
\end{ex}
% ===================================================================
\begin{ex}
	Một học sinh làm thí nghiệm đo nhiệt hoá hơi riêng của nước tại nhà như sau. Đổ \SI{380}{\gram} nước ở nhiệt độ phòng \SI{20}{\celsius} vào đun sôi trong một ấm điện chuyên dụng như hình vẽ. Các thông số kĩ thuật của ấm điện được cho như bảng 1.
	\begin{center}
		\includegraphics[scale=0.4]{figs/G12Y24B7-1}
	\end{center}
	Ngoài ra, học sinh còn dùng cân điện tử để cân lượng nước còn lại trong ấm và dùng đồng hồ để đo thời gian đun. Khi nước sôi ở \SI{100}{\celsius} thì học sinh mở nắp ấm cho hơi nước dễ bay ra và bắt đầu ghi lại số liệu khi lượng nước còn lại trong ấm là \SI{350}{\gram}. Đồ thị sự phụ thuộc của khối lượng nước $m$ còn lại trong ấm vào thời gian đun $\tau$ như đồ thị bên dưới.
	\begin{center}
		\includegraphics[scale=0.4]{figs/G12Y24B7-2}
	\end{center}
	Biết rằng khi nước chưa sôi thì hiệu suất đun nước của ấm bằng \SI{96}{\percent} còn khi nước sôi thì hiệu suất ấm đun giảm xuống còn \SI{92}{\percent}, nhiệt dung riêng của nước là \SI{4200}{\joule/\left(\kilogram\cdot\kelvin\right)}.
	\choiceTF[t]
	{\True Nếu khi nước sôi không mở nắp ấm thì thời gian để đun cạn nước trong ấm sẽ tăng lên}
	{Độ hụt khối lượng của nước trong ấm sau mỗi giây bằng \SI{0.34}{\gram/\second}. \textit{(Làm tròn kết quả đến hai chữ số sau dấu phẩy thập phân)}}
	{Nhiệt hoá hơi riêng của nước trong thí nghiệm này bằng \SI{2.33}{\mega\joule/\kilogram}. \textit{(Làm tròn kết quả đến hai chữ số sau dấu phẩy thập phân)}}
	{Tổng thời gian đun nước đến khi cạn bằng \SI{556.39}{\second}. \textit{(Làm tròn kết quả đến hai chữ số sau dấu phẩy thập phân)}}
	\loigiai{
\begin{enumerate}[label=\alph*)]
	\item Đúng. Nếu khi nước sôi không mở nắp ấm thì thời gian để đun cạn nước trong ấm sẽ tăng lên do đậy nắp ấm làm cho hơi nước thoát ra ngoài khó hơn nên việc hoá hơi gặp khó khăn hơn.
	\item Sai. Từ đồ thị ta thấy:\\
	Khi $m_1=\SI{350}{\gram}$ thì $\tau_1=\SI{0}{\second}$.\\
	Khi $m_2=\SI{200}{\gram}$ thì $\tau_2=\SI{220}{\second}$.\\
	Độ hụt khối lượng của nước trong ấm sau mỗi giây bằng: $\dfrac{\Delta m}{\Delta \tau}=\dfrac{350-200}{220} \approx \SI{0.68}{\gram/\second}$.
	\item Sai.\\
	Năng lượng ấm toả ra: $W=\calP \cdot\left(\tau_2-\tau_1\right)=1700\cdot220= \SI{374000}{\joule}$.\\
	Năng lượng nước thu vào trong quá trình bay hơi: $Q=H\cdot Q=0,92\cdot374000=\SI{344080}{\joule}$.\\
	Nhiệt hóa hơi riêng của nước:
	$$
	L=\dfrac{Q}{m_1-m_2}=\dfrac{344080}{0,35-0,2} \approx \SI{2.29}{\mega\joule/\kilogram},
	$$
	\item Sai.\\
	Nhiệt lượng \SI{380}{\gram} nước thu vào để tăng đến nhiệt độ sôi:
	$$
Q_1=mc \Delta T=0,38\cdot4200 \cdot\left(100-20\right)=\SI{127680}{\joule}.
	$$
	Thời gian bếp đun từ lúc sôi đến khi bay hơi hết: $t_1=\dfrac{W_1}{\calP_1}=\dfrac{\dfrac{Q_1}{H_1}}{\calP_1}=\dfrac{\dfrac{127680}{0,96}}{2500}=\SI{53.2}{\second}$.\\
	Nhiệt lượng \SI{380}{\gram} nước thu vào để bay hơi hoàn toàn ở nhiệt độ sôi là:
	$$
Q_2=mL=0,38\cdot2293867=\SI{871669}{\joule}.
	$$
	Thời gian bếp đun từ lúc sôi đến khi bay hơi hết: $t_2=\dfrac{W_2}{\calP_2}=\dfrac{\dfrac{Q_2}{H_2}}{\mathrm{P}_2}=\dfrac{\dfrac{871669}{0,92}}{1700}=\SI{557.3}{\second}$.\\ Tổng thời gian đun nước: $t=t_1+t_2=53,2+557,3=\SI{610.5}{\second}$.
\end{enumerate}
	}
\end{ex}
% ===================================================================
\begin{ex}
	Dùng bếp điện để đun một ấm nhôm khối lượng $\SI{600}{\gram}$ đựng $\SI{1.5}{\text{lít}}$ nước ở nhiệt độ $\SI{20}{\celsius}$. Sau $\SI{35}{\minute}$ đã có $\SI{20}{\percent}$ lượng nước trong ấm hoá hơi ở nhiệt độ sôi $\SI{100}{\celsius}$. Biết có $\SI{60}{\percent}$ nhiệt lượng mà bếp toả ra được dùng vào việc đun ấm nước. Cho nhiệt dung riêng của nhôm là $\SI{880}{\joule/\left(\kilogram\cdot\kelvin\right)}$, của nước là $\SI{4200}{\joule/\left(\kilogram\cdot\kelvin\right)}$, nhiệt hoá hơi riêng của nước ở nhiệt độ sôi $\SI{100}{\celsius}$ là $\SI{2.26E6}{\joule/\kilogram}$ và khối lượng riêng của nước là $\SI{1}{\kilogram/\text{lít}}$.
	\choiceTF[t]
	{Nhiệt lượng cần thiết để đun ấm nước từ $\SI{20}{\celsius}$ đến $\SI{100}{\celsius}$ là $\SI{504000}{\joule}$}
	{\True Khối lượng nước đã hoá hơi là $\SI{0.03}{\kilogram}$}
	{\True Nhiệt lượng mà bếp điện cung cấp để đun nước đến khi sôi là $\SI{910.4}{\kilo\joule}$}
	{Nhiệt lượng trung bình mà bếp điện cung cấp cho ấm nước trong mỗi giây là $\SI{582.97}{\joule}$}
	\loigiai{\begin{enumerate}[label=\alph*)]
			\item Sai. Nhiệt lượng cần thiết để đun sôi ấm nước là $\SI{546240}{\joule}$.
			\item Đúng.
			\item Đúng.
			\item Sai. Nhiệt lượng trung bình mà bếp điện cung cấp cho ấm nước trong mỗi giây là $\SI{971.6}{\joule}$.
	\end{enumerate}}
\end{ex}

\subsection{CÂU TRẮC NGHIỆM TRẢ LỜI NGẮN}
\setcounter{ex}{0}
\textit{Thí sinh trả lời từ câu 1 đến câu 6.}
% ===================================================================
\begin{ex}
Giả sử rằng các tuabin ở nhà máy nhiệt điện đã được nâng cấp, dẫn đến sự cải thiện về hiệu suất $\SI{3.32}{\percent}$. Biết rằng trước khi nâng cấp thì hiệu suất của nhà máy điện là $\SI{36}{\percent}$, nhiệt lượng truyền vào động cơ trong một ngày vẫn không đổi và bằng $\SI{2.5E14}{\joule}$. Có thêm bao nhiêu lượng điện năng được sản xuất trong 1 ngày nhờ vào sự nâng cấp trên \textit{(tính theo đơn vị  $\SI{E12}{\joule}$)}?
	\loigiai{$$\Delta A=Q_1\Delta H=\left(\SI{2.5E14}{\joule}\right)\cdot\left(\SI{3.32}{\percent}\right)=\SI{8.3E12}{\joule}.$$
	}
\end{ex}
% ===================================================================
\begin{ex}
	Người ta thả một miếng đồng khối lượng $\SI{0.5}{\kilogram}$ vào $\SI{500}{\gram}$ nước. Miếng đồng nguội đi từ $\SI{80}{\celsius}$ xuống $\SI{20}{\celsius}$. Hỏi nước nóng lên thêm bao nhiêu $\si{\celsius}$ \textit{(làm tròn đến 2 số thập phân)}? Biết nhiệt dung riêng của đồng là $\SI{380}{\joule/\left(\kilogram\cdot\kelvin\right)}$, nhiệt dung riêng của nước là $\SI{4200}{\joule/\left(\kilogram\cdot\kelvin\right)}$.
	\loigiai{$$\Delta t=\dfrac{m_\text{đ}c_\text{đ}\left(80-20\right)}{m_\text{n}c_\text{n}}\approx\SI{5.43}{\celsius}.$$}
\end{ex}
% ===================================================================
\begin{ex}
Một ấm điện có công suất $\SI{1000}{\watt}$. Tính thời gian cần thiết để đun sôi $\SI{0.5}{\text{lít}}$ nước có nhiệt độ ban đầu là $\SI{20}{\celsius}$ ở áp suất tiêu chuẩn theo đơn vị phút. Bỏ qua sự trao đổi nhiệt với vỏ ấm và môi trường. Cho hiệu suất ấm đun là $\SI{40}{\percent}$. Biết nhiệt dung riêng của nước là $\SI{4200}{\joule/\left(\kilogram\cdot\kelvin\right)}$ và khối lượng riêng của nước là $\SI{1}{\kilogram/\liter}$.
	\loigiai{$$t=\dfrac{mc\Delta t}{H\calP}\approx\SI{7}{\minute}.$$
	}
\end{ex}
% ===================================================================
\begin{ex}
	Tính nhiệt lượng cần cung cấp \textit{(theo đơn vị $\si{\kilo\joule})$} cho $\SI{10}{\kilogram}$ nước ở $\SI{30}{\celsius}$ chuyển thành hơi ở $\SI{100}{\celsius}$. Cho biết nhiệt dung riêng của nước $\SI{4180}{\joule/\left(\kilogram\cdot\kelvin\right)}$ và nhiệt hoá hơi riêng của nước là $\SI{2.3E6}{\joule/\kilogram}$.
	\loigiai{$$Q=mc\Delta t+mL=\SI{25926}{\kilo\joule}.$$
	}
\end{ex}
% ===================================================================
\begin{ex}
	\immini{
	10 viên nước đá được dùng để làm lạnh cốc nước soda có khối lượng $\SI{0.25}{\kilogram}$, mỗi viên đá có khối lượng $\SI{6}{\gram}$. Ban đầu, nước soda trong cốc có nhiệt độ $\SI{20}{\celsius}$. Xác định nhiệt độ cốc nước khi đá tan hết theo đơn vị $\si{\celsius}$ và làm tròn đến 2 chữ số thập phân. Biết rằng nhiệt dung riêng của nước là $\SI{4186}{\joule/\left(\kilogram\cdot\kelvin\right)}$, nhiệt nóng chảy riêng của nước đá là $\SI{3.34E5}{\joule/\kilogram}$. Bỏ qua sự trao đổi nhiệt với cốc và môi trường bên ngoài.
}
{
\includegraphics[scale=0.25]{figs/VN12-Y24-PH-SYL-008-1}
}
	\loigiai{Khi có cân bằng nhiệt, tổng nhiệt lượng trao đổi trong hệ bằng 0:
		$$10m_\text{đ}\lambda+m_\text{n}c_\text{n}\left(t_\text{cb}-t_0\right)=0$$
		$$\Rightarrow t_\text{cb}\approx\SI{0.69}{\celsius}.$$}
\end{ex}
% ===================================================================
\begin{ex}
	Năm 1986, một tảng băng khổng lồ đã tách ra khỏi thềm băng Ross ở Nam Cực. Tảng băng có dạng gần như hình hộp chữ nhật với chiều dài $\SI{160}{\kilo\meter}$, chiều rộng $\SI{40}{\kilo\meter}$ và dày $\SI{250}{\meter}$. Khối lượng riêng của băng là $\SI{917}{\kilogram/\meter^3}$, nhiệt nóng chảy riêng của băng là $\SI{3.34E5}{\joule/\kilogram}$. Chỉ riêng ánh sáng Mặt Trời thì phải mất bao nhiêu năm để làm tan chảy được lớp băng dày như thế \textit{(làm tròn đến 2 chữ số thập phân)}. Cho rằng công suất toả nhiệt trung bình của Mặt Trời là $\SI{100}{\watt/\meter^2}$ và Mặt Trời chiếu sáng $\SI{12}{\hour}$ mỗi ngày.
	
	\loigiai{Nhiệt lượng cần cung cấp để làm tan khối băng:
		$$Q=\rho V\lambda=Sh\rho\lambda.$$
		Nhiệt lượng do Mặt Trời cung cấp:
		$$Q=\calP t$$
		Thời gian cần để băng tan:
		$$t=\dfrac{\rho h\lambda}{\calP}=\SI{212693}{\hour}.$$
		Mỗi ngày trung bình Mặt Trời chiếu sáng $\SI{12}{\hour}$ nên nếu chỉ riêng Mặt Trời chiếu sáng thì phải mất:
		$$\dfrac{t}{12\cdot 365}\approx\SI{48.56}{\text{năm}}.$$}
\end{ex}